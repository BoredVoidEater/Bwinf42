\documentclass[a4paper,10pt,ngerman]{scrartcl}
\usepackage{babel}
\usepackage[T1]{fontenc}
\usepackage[utf8x]{inputenc}
\usepackage[a4paper,margin=2.5cm,footskip=0.5cm]{geometry}

% Die nächsten vier Felder bitte anpassen:
\newcommand{\Aufgabe}{Aufgabe 4: Nandu} % Aufgabennummer und Aufgabennamen angeben
\newcommand{\TeamId}{00112}                       % Team-ID aus dem PMS angeben
\newcommand{\TeamName}{10m Dürer gym}                 % Team-Namen angeben
\newcommand{\Namen}{Finn Degen}           % Namen der Bearbeiter/-innen dieser Aufgabe angeben


% tables
\usepackage{tabularx}

% Kopf- und Fußzeilen
\usepackage{scrlayer-scrpage, lastpage}
\setkomafont{pageheadfoot}{\large\textrm}
\lohead{\Aufgabe}
\rohead{Team-ID: \TeamId}
\cfoot*{\thepage{}/\pageref{LastPage}}

% Position des Titels
\usepackage{titling}
\setlength{\droptitle}{-1.0cm}

% Für mathematische Befehle und Symbole
\usepackage{amsmath}
\usepackage{amssymb}

% Für Bilder
\usepackage{graphicx}

% Für Algorithmen
\usepackage{algpseudocode}
\algnewcommand{\algorithmicand}{\textbf{ and }}
\algnewcommand{\algorithmicor}{\textbf{ or }}
\algnewcommand{\OR}{\algorithmicor}
\algnewcommand{\AND}{\algorithmicand}
\algnewcommand{\var}{\texttt}

% Für Quelltext
\usepackage{listings}
\usepackage{xcolor}
\definecolor{mygreen}{rgb}{0,0.6,0}
\definecolor{mygray}{rgb}{0.5,0.5,0.5}
\definecolor{mymauve}{rgb}{0.58,0,0.82}
\lstset{
  keywordstyle=\color{blue},commentstyle=\color{mygreen},
  stringstyle=\color{mymauve},rulecolor=\color{black},
  basicstyle=\footnotesize\ttfamily,numberstyle=\tiny\color{mygray},
  captionpos=b, % sets the caption-position to bottom
  keepspaces=true, % keeps spaces in text
  numbers=left, numbersep=5pt, showspaces=false,showstringspaces=true,
  showtabs=false, stepnumber=2, tabsize=2, title=\lstname
}
\lstdefinelanguage{JavaScript}{ % JavaScript ist als einzige Sprache noch nicht vordefiniert
  keywords={break, case, catch, continue, debugger, default, delete, do, else, finally, for, function, if, in, instanceof, new, return, switch, this, throw, try, typeof, var, void, while, with},
  morecomment=[l]{//},
  morecomment=[s]{/*}{*/},
  morestring=[b]',
  morestring=[b]",
  sensitive=true
}
\lstset{literate=%
  {Ö}{{\"O}}1
  {Ä}{{\"A}}1
  {Ü}{{\"U}}1
  {ß}{{\ss}}1
  {ü}{{\"u}}1
  {ä}{{\"a}}1
  {ö}{{\"o}}1
}

%Für Beispiele im Text
\usepackage{amsthm}
\usepackage[framemethod=tikz]{mdframed}
\newmdtheoremenv[
  hidealllines=true,
  leftline=true,
  innertopmargin=0pt,
  innerbottommargin=0pt,
  linewidth=4pt,
  linecolor=gray!40,
  innerrightmargin=0pt,
  innertopmargin=-6pt,
]{examplei}{Beispiel}

% Diese beiden Pakete müssen zuletzt geladen werden
\usepackage{hyperref} % Anklickbare Links im Dokument
\usepackage[nameinlink]{cleveref}

% Daten für die Titelseite
\title{\textbf{\Huge\Aufgabe}}
\author{\LARGE Team-ID: \LARGE \TeamId \\\\
	    \LARGE Team-Name: \LARGE \TeamName \\\\
	    \LARGE Bearbeiter/-innen dieser Aufgabe: \\ 
	    \LARGE \Namen\\\\}
\date{\LARGE\today}

\begin{document}

\maketitle
\tableofcontents

\vspace{0.5cm}


\section{Lösungsidee}
\label{Lösungsidee}\label{sec:loesungsidee}
\subsection{Identifikation der Blocktypen}
Um die Aufgabe zu lösen müssen wir erstmal definieren, was die einzelnen Blöcke genau machen.
\begin{itemize}
  \setlength\itemsep{0.5cm}
  \item \emph{weißer Block}: \resizebox{5cm}{!}{\begin{tabular}{ |c|c|c|c| }\hline IN1 & IN2 & OUT1 & OUT2 \\ \hline 0 & 0 & 1 & 1 \\  0 & 1 & 1 & 1 \\ 1 & 0 & 1 & 1 \\ 1 & 1 & 0 & 0 \\ \hline    \end{tabular}} $\Rightarrow$ \begin{minipage}{0.4\textwidth}Wenn wir IN1, IN2 und ein beliebiges OUT betrachten, erinnert es stark an eine AND Wahrheitstabelle, nur ist OUT eben umgekehrt $\Rightarrow$ OUT1 = OUT2 = NOT(AND(IN1,IN2)) = NAND(IN1,IN2) \end{minipage}
  \item \emph{roter Block}: \hspace{1.375cm} \resizebox{3.75cm}{!}{\begin{tabular}{ |c|c|c| }\hline IN1 & OUT1 & OUT2 \\ \hline 0 & 1 & 1 \\  1 & 0 & 0 \\ \hline    \end{tabular}} $\Rightarrow$ \begin{minipage}{0.4\textwidth}Wenn wir IN1 und ein beliebiges OUT betrachten, sieht man den NOT zusammenhand $\Rightarrow$ OUT1 = OUT2 = NOT(IN1)\end{minipage}
  \item \emph{blauer Block}: \resizebox{5cm}{!}{\begin{tabular}{ |c|c|c|c| }\hline IN1 & IN2 & OUT1 & OUT2 \\ \hline 0 & 0 & 0 & 0 \\  0 & 1 & 0 & 1 \\ 1 & 0 & 1 & 0 \\ 1 & 1 & 1 & 1 \\ \hline    \end{tabular}} $\Rightarrow$ \begin{minipage}{0.4\textwidth}Wenn wir IN1, IN2, OUT1 und OUT2 betrachten, sieht man das 1 und 2 jeweils gleich sind $\Rightarrow$ OUT1 = IN1; OUT2 = IN2 \end{minipage}
\end{itemize}
\subsection{Iteratives Berechnen der Zustände}
Nun brauchen wir nur noch eine Methode, diese Blöcke in Gittern einzulesen und die Zustände zu berechnen.
Dazu nehmen wir ein 2D Array \emph{CALC\textunderscore TABLE}, in dem wir die Zustände für jede Koordinate des Gitters speichern.\\
\begin{examplei}
Für ein 3x3 Gitter sieht das dann so aus:\\
CALC\textunderscore TABLE = [ [0,0,0], [0,0,0], [0,0,0] ]
\end{examplei}
Als ersten Schritt finden wir dann die Koordinaten unserer Input-Licht Blöcke (Q) und setzen diese in \emph{CALC\textunderscore TABLE} auf die gewünschten Wahrheitswerte.
Dann gehen wir durch jede Koordinate des Gitters: erst link nach rechts, dann oben nach unten und setzen den geradigen Wahrheitswert zu der spezifischen Block Funktion (siehe oben + Q + L nimmt Input von Reihe davor auf). Die Inputs nehmen wir von den Koordinaten eine Reihe davor. 
Um die richtige Blockfunktion zu finden, brauchen wir noch ein 2D Array \emph{BLOCK\textunderscore TABLE}, in dem wir die Blöcke (W,R,r,B,Q,L) speichern.\\
\begin{examplei}
Für ein 3x3 Gitter mit einem eingeschalteten Licht in der Mitte oben und einem blauem Block darunter sieht das dann so aus:\\
Iteration 1:\\
CALC\textunderscore TABLE = [ [0,1,0], [0,,0], [0,0,0] ]\\
BLOCK\textunderscore TABLE = [ [0,Q,0], [B,B,0], [0,0,0] ]\\
Iteration 2:\\
CALC\textunderscore TABLE = [ [0,1,0], [0,1,0], [0,0,0] ]\\
BLOCK\textunderscore TABLE = [ [0,Q,0], [B,B,0], [L,L,0] ]\\
\end{examplei}
Nach der letzten Iteration haben wir dann die Lösung für das Gitter in \emph{CALC\textunderscore TABLE} gespeichert und können dann die Zustände aller Koordinaten ausgeben, die einen Output-Licht Block (L) haben.\\



\clearpage
\section{Umsetzung}\label{sec:umsetzung}
Der folgende Pseudocode beschreibt den Lösungsalgorithmus für die Aufgabe.\\
Zuerst werden folgende Variablen initialisiert:
\begin{algorithmic}
    \State $\var{CALC\textunderscore TABLE} \gets \var{2D Array mit der der Größe des Gitters}$
    \State $\var{BLOCK\textunderscore TABLE} \gets \Call{AufgabeEinlesenZuBlockArray}$
    \State $\var{INPUT\textunderscore COORDINATES} \gets \Call{FindeInputKoordinaten}$
    \For{$\var{Koordinate} \in \var{INPUT\textunderscore COORDINATES}$}
        \State $\var{CALC\textunderscore TABLE[Koordinate]} \gets \var{Wahrheitswert des Input-Lichts}$
    \EndFor
    \State $\var{OUTPUT\textunderscore COORDINATES} \gets \Call{FindeOutputKoordinaten}$
    \\
\end{algorithmic}
Dann wird das Gitter iterativ durchgegangen und die Zustände berechnet:
\begin{algorithmic}
    \For{$\var{Zeile} \in \var{CALC\textunderscore TABLE}$}
        \For{$\var{Koordinate} \in \var{Zeile}$}
            \State $\var{CALC\textunderscore TABLE[Koordinate]} \gets \Call{BerechneBlock}{\var{CALC\textunderscore TABLE[Koordinate]}, \var{CALC\textunderscore TABLE}, \var{Koordinate}}$
        \EndFor
    \EndFor
\end{algorithmic}
Die Methode \emph{BerechneBlock} berechnet den Zustand einer Koordinate anhand der Blockfunktion und den Inputs. Die Blockfunktion ordnet sie anhand des geradigen und des letzen Blockcharakters aus \emph{BLOCK\textunderscore TABLE} zu, da die großen Blöcke ja 2 Einheiten lang sind. Die Inputs nimmt sie von den Koordinaten eine Reihe davor. Sie wird später in \cref{sec:quellcode} noch genauer erläutert.\\\\
Zum Schluss werden die Zustände der Output-Lichter ausgegeben:
\begin{algorithmic}
    \For{$\var{Koordinate} \in \var{OUTPUT\textunderscore COORDINATES}$}
        \State $\Call{Ausgabe}{\var{CALC\textunderscore TABLE[Koordinate]}}$
    \EndFor
\end{algorithmic}


\section{Beispiele}
Im folgenden wird das Programm mit allen Beispielaufgaben ausgeführt
\paragraph{Beispiel 1} \mbox{} \\
\begin{lstlisting}[frame=tb]
Q1: False Q2: False L1: True L2: True 
Q1: False Q2: True L1: True L2: True 
Q1: True Q2: False L1: True L2: True 
Q1: True Q2: True L1: False L2: False
\end{lstlisting}


\paragraph{Beispiel 2} \mbox{} \\
\begin{lstlisting}[frame=tb]
Q1: False Q2: False L1: False L2: True
Q1: False Q2: True L1: False L2: True
Q1: True Q2: False L1: False L2: True
Q1: True Q2: True L1: True L2: False
\end{lstlisting}

\paragraph{Beispiel 3} \mbox{} \\
\begin{lstlisting}[frame=tb]
Q1: False Q2: False Q3: False L1: True L2: False L3: False L4: True
Q1: False Q2: False Q3: True L1: True L2: False L3: False L4: False
Q1: False Q2: True Q3: False L1: True L2: False L3: True L4: True
Q1: False Q2: True Q3: True L1: True L2: False L3: True L4: False
Q1: True Q2: False Q3: False L1: False L2: True L3: False L4: True
Q1: True Q2: False Q3: True L1: False L2: True L3: False L4: False
Q1: True Q2: True Q3: False L1: False L2: True L3: True L4: True
Q1: True Q2: True Q3: True L1: False L2: True L3: True L4: false
\end{lstlisting}

\paragraph{Beispiel 4} \mbox{} \\
\begin{lstlisting}[frame=tb]
Q1: False Q2: False Q3: False Q4: False L1: False L2: False
Q1: False Q2: False Q3: False Q4: True L1: False L2: False
Q1: False Q2: False Q3: True Q4: False L1: False L2: True
Q1: False Q2: False Q3: True Q4: True L1: False L2: False
Q1: False Q2: True Q3: False Q4: False L1: True L2: False
Q1: False Q2: True Q3: False Q4: True L1: True L2: False
Q1: False Q2: True Q3: True Q4: False L1: True L2: True
Q1: False Q2: True Q3: True Q4: True L1: True L2: False
Q1: True Q2: False Q3: False Q4: False L1: False L2: False
Q1: True Q2: False Q3: False Q4: True L1: False L2: False
Q1: True Q2: False Q3: True Q4: False L1: False L2: True
Q1: True Q2: False Q3: True Q4: True L1: False L2: False
Q1: True Q2: True Q3: False Q4: False L1: False L2: False
Q1: True Q2: True Q3: False Q4: True L1: False L2: False
Q1: True Q2: True Q3: True Q4: False L1: False L2: True
Q1: True Q2: True Q3: True Q4: True L1: False L2: False
\end{lstlisting}

\paragraph{Beispiel 5} 
Achtung: Hier habe ich Absätze machen müssen, weil der Platz sonst nicht reicht 
\begin{lstlisting}[frame=tb,breaklines=true]
Q1: False Q2: False Q3: False Q4: False Q5: False Q6: False L1: False L2: False L3: False L4: True L5: False
Q1: False Q2: False Q3: False Q4: False Q5: False Q6: True L1: False L2: False L3: False L4: True L5: False
Q1: False Q2: False Q3: False Q4: False Q5: True Q6: False L1: False L2: False L3: False L4: True L5: True
Q1: False Q2: False Q3: False Q4: False Q5: True Q6: True L1: False L2: False L3: False L4: True L5: True
Q1: False Q2: False Q3: False Q4: True Q5: False Q6: False L1: False L2: False L3: True L4: False L5: False
Q1: False Q2: False Q3: False Q4: True Q5: False Q6: True L1: False L2: False L3: True L4: False L5: False
Q1: False Q2: False Q3: False Q4: True Q5: True Q6: False L1: False L2: False L3: False L4: True L5: True
Q1: False Q2: False Q3: False Q4: True Q5: True Q6: True L1: False L2: False L3: False L4: True L5: True
Q1: False Q2: False Q3: True Q4: False Q5: False Q6: False L1: False L2: False L3: False L4: True L5: False
Q1: False Q2: False Q3: True Q4: False Q5: False Q6: True L1: False L2: False L3: False L4: True L5: False
Q1: False Q2: False Q3: True Q4: False Q5: True Q6: False L1: False L2: False L3: False L4: True L5: True
Q1: False Q2: False Q3: True Q4: False Q5: True Q6: True L1: False L2: False L3: False L4: True L5: True
Q1: False Q2: False Q3: True Q4: True Q5: False Q6: False L1: False L2: False L3: True L4: False L5: False
Q1: False Q2: False Q3: True Q4: True Q5: False Q6: True L1: False L2: False L3: True L4: False L5: False
Q1: False Q2: False Q3: True Q4: True Q5: True Q6: False L1: False L2: False L3: False L4: True L5: True
Q1: False Q2: False Q3: True Q4: True Q5: True Q6: True L1: False L2: False L3: False L4: True L5: True
Q1: False Q2: True Q3: False Q4: False Q5: False Q6: False L1: False L2: False L3: False L4: True L5: False
Q1: False Q2: True Q3: False Q4: False Q5: False Q6: True L1: False L2: False L3: False L4: True L5: False
Q1: False Q2: True Q3: False Q4: False Q5: True Q6: False L1: False L2: False L3: False L4: True L5: True
Q1: False Q2: True Q3: False Q4: False Q5: True Q6: True L1: False L2: False L3: False L4: True L5: True
Q1: False Q2: True Q3: False Q4: True Q5: False Q6: False L1: False L2: False L3: True L4: False L5: False
Q1: False Q2: True Q3: False Q4: True Q5: False Q6: True L1: False L2: False L3: True L4: False L5: False
Q1: False Q2: True Q3: False Q4: True Q5: True Q6: False L1: False L2: False L3: False L4: True L5: True
Q1: False Q2: True Q3: False Q4: True Q5: True Q6: True L1: False L2: False L3: False L4: True L5: True
Q1: False Q2: True Q3: True Q4: False Q5: False Q6: False L1: False L2: False L3: False L4: True L5: False 
Q1: False Q2: True Q3: True Q4: False Q5: False Q6: True L1: False L2: False L3: False L4: True L5: False
Q1: False Q2: True Q3: True Q4: False Q5: True Q6: False L1: False L2: False L3: False L4: True L5: True
Q1: False Q2: True Q3: True Q4: False Q5: True Q6: True L1: False L2: False L3: False L4: True L5: True
Q1: False Q2: True Q3: True Q4: True Q5: False Q6: False L1: False L2: False L3: True L4: False L5: False
Q1: False Q2: True Q3: True Q4: True Q5: False Q6: True L1: False L2: False L3: True L4: False L5: False
Q1: False Q2: True Q3: True Q4: True Q5: True Q6: False L1: False L2: False L3: False L4: True L5: True
Q1: False Q2: True Q3: True Q4: True Q5: True Q6: True L1: False L2: False L3: False L4: True L5: True
Q1: True Q2: False Q3: False Q4: False Q5: False Q6: False L1: True L2: False L3: False L4: True L5: False
Q1: True Q2: False Q3: False Q4: False Q5: False Q6: True L1: True L2: False L3: False L4: True L5: False
Q1: True Q2: False Q3: False Q4: False Q5: True Q6: False L1: True L2: False L3: False L4: True L5: True
Q1: True Q2: False Q3: False Q4: False Q5: True Q6: True L1: True L2: False L3: False L4: True L5: True
Q1: True Q2: False Q3: False Q4: True Q5: False Q6: False L1: True L2: False L3: True L4: False L5: False
Q1: True Q2: False Q3: False Q4: True Q5: False Q6: True L1: True L2: False L3: True L4: False L5: False
Q1: True Q2: False Q3: False Q4: True Q5: True Q6: False L1: True L2: False L3: False L4: True L5: True
Q1: True Q2: False Q3: False Q4: True Q5: True Q6: True L1: True L2: False L3: False L4: True L5: True
Q1: True Q2: False Q3: True Q4: False Q5: False Q6: False L1: True L2: False L3: False L4: True L5: False
Q1: True Q2: False Q3: True Q4: False Q5: False Q6: True L1: True L2: False L3: False L4: True L5: False
Q1: True Q2: False Q3: True Q4: False Q5: True Q6: False L1: True L2: False L3: False L4: True L5: True
Q1: True Q2: False Q3: True Q4: False Q5: True Q6: True L1: True L2: False L3: False L4: True L5: True
Q1: True Q2: False Q3: True Q4: True Q5: False Q6: False L1: True L2: False L3: True L4: False L5: False
Q1: True Q2: False Q3: True Q4: True Q5: False Q6: True L1: True L2: False L3: True L4: False L5: False
Q1: True Q2: False Q3: True Q4: True Q5: True Q6: False L1: True L2: False L3: False L4: True L5: True
Q1: True Q2: False Q3: True Q4: True Q5: True Q6: True L1: True L2: False L3: False L4: True L5: True
Q1: True Q2: True Q3: False Q4: False Q5: False Q6: False L1: True L2: False L3: False L4: True L5: False
Q1: True Q2: True Q3: False Q4: False Q5: False Q6: True L1: True L2: False L3: False L4: True L5: False
Q1: True Q2: True Q3: False Q4: False Q5: True Q6: False L1: True L2: False L3: False L4: True L5: True
Q1: True Q2: True Q3: False Q4: False Q5: True Q6: True L1: True L2: False L3: False L4: True L5: True
Q1: True Q2: True Q3: False Q4: True Q5: False Q6: False L1: True L2: False L3: True L4: False L5: False
Q1: True Q2: True Q3: False Q4: True Q5: False Q6: True L1: True L2: False L3: True L4: False L5: False
Q1: True Q2: True Q3: False Q4: True Q5: True Q6: False L1: True L2: False L3: False L4: True L5: True
Q1: True Q2: True Q3: False Q4: True Q5: True Q6: True L1: True L2: False L3: False L4: True L5: True
Q1: True Q2: True Q3: True Q4: False Q5: False Q6: False L1: True L2: False L3: False L4: True L5: False
Q1: True Q2: True Q3: True Q4: False Q5: False Q6: True L1: True L2: False L3: False L4: True L5: False
Q1: True Q2: True Q3: True Q4: False Q5: True Q6: False L1: True L2: False L3: False L4: True L5: True
Q1: True Q2: True Q3: True Q4: False Q5: True Q6: True L1: True L2: False L3: False L4: True L5: True
Q1: True Q2: True Q3: True Q4: True Q5: False Q6: False L1: True L2: False L3: True L4: False L5: False
Q1: True Q2: True Q3: True Q4: True Q5: False Q6: True L1: True L2: False L3: True L4: False L5: False
Q1: True Q2: True Q3: True Q4: True Q5: True Q6: False L1: True L2: False L3: False L4: True L5: True
Q1: True Q2: True Q3: True Q4: True Q5: True Q6: True L1: True L2: False L3: False L4: True L5: True
\end{lstlisting}

\paragraph{Eigenes Beispiel 1} Dieses Beispiel zeigt, dass auch Input-Lichter die nicht in der ersten Reihe sind funktionieren (In den BWINF Beispielen waren diese immer in der ersten Reihe)
\begin{lstlisting}[frame=tb]
Input:
3 4
X  X X  
X  X  Q1 
X  B  B  
X  X  L1 

Output:
Q1: False L1: False
Q1: True L1: True
\end{lstlisting}

\paragraph{Eigenes Beispiel 2} Dieses Beispiel zeigt, dass Licht nicht über mehrere Blöcke hinweg leuchten kann. Ich habe mich bewusst dagegen entschieden, da der blaue Block sonst keinen Sinn machen würde.
\begin{lstlisting}[frame=tb]
Input:
3 4
X  X  Q1  
X  X  X 
X  B  B  
X  X  L1 

Output:
Q1: False L1: False
Q1: True L1: False
\end{lstlisting}

\section{Quellcode}
\label{Quellcode}\label{sec:quellcode}
Wir fangen mit der Initialisierung der Variablen wie in \cref{sec:umsetzung} an:
\begin{lstlisting}[language=Python]
table: list[list[str]] = []  # Pseudocode: BLOCK_TABLE
for row in raw_data:  # Pseudocode: AufgabeEinlesenZuBlockArray
    table.append(row.split())

# initialize Qs and Ls
num_q = 0
indecies_q = []
indecies_l = []
for row_index, _ in enumerate(table):
    for column_index, element in enumerate(table[row_index]):  # first row
        if element.startswith("Q"):
            num_q += 1
            indecies_q.append((row_index, column_index))
        elif element.startswith("L"):
            indecies_l.append((row_index, column_index))
\end{lstlisting}
Da wir alle möglichen Input-Licht Kombinationen ausgeben müssen, brauchen wir eine Methode, die uns alle Kombinationen von 0 und 1 mit einer bestimmten Länge ausgibt. Dazu benutzen wir die \emph{itertools} Bibliothek.
\begin{lstlisting}[language=Python]
possible_q_combinations = list(itertools.product([False, True], repeat=num_q))
\end{lstlisting}
Nun implementieren wir die Haupt Algorithmus, dies ist der selbe wie in \cref{sec:umsetzung} nur wird er n mal für alle permutationen der Input-Lichter ausgeführt
\begin{lstlisting}[language=Python]
for possible_q_combination in possible_q_combinations:
  # this table saves the booleans for the light outputs
  # Psuedocode: CALC_TABLE
  calculation_table = [[False for _ in range(len(table[0]))] for _ in range(len(table))]

  # print Qs
  for num_of_possible_q, possible_q_value in enumerate(possible_q_combination):
      # start printing table
      print(f"Q{num_of_possible_q+1}: {possible_q_value}", end=" ")
      # add the current value of true or false for
      # each Q so that we have all combinations in the end
      calculation_table[indecies_q[num_of_possible_q][0]][
          indecies_q[num_of_possible_q][1]
      ] = possible_q_combination[num_of_possible_q]

  # go trough each row
  # and compute the bool values for the light outputs
  for row_index, _ in enumerate(table):
      last_element = None # last element has to be reset, else it will think that blocks on the edge span over 2 rows
      for column_index, element in enumerate(table[row_index]):
          # Pseudocode: BerechneBlock
          last_element = calculate_element(element, last_element, calculation_table, row_index, column_index)

  # print Ls
  for num_current_l, (row_index, column_index) in enumerate(indecies_l):
      print(
          f"L{num_current_l+1}: {calculation_table[row_index][column_index]}",
          end=" ",
      )

\end{lstlisting}
Nun schauen wir uns nochmal die Methode \emph{calculate\textunderscore element} an, die die neuen Zustände der Blöcke berechnet.
\begin{lstlisting}[language=Python]
def calculate_element(element: str, last_element: str, calculation_table, row_index, column_index):
"""Berechnet den Wahrheitswehrt für das gegebene Element mit der zusätzlichen Information des letzten Elements. Speichert den Wahrheitswert in calculation_table und gibt das neue letzte Element zurück

:param element: Das aktuelle Element
:param last_element: Das letzte Element
:param calculation_table: Die Tabelle mit den Wahrheitswerten. Wird direkt verändert
:param row_index: Die aktuelle Zeile
:param column_index: Die aktuelle Spalte
:returns: Das neue letzte Element
"""
if last_element:
    match element:
        case "W":
            if last_element == "W":
                calculation_table[row_index][column_index] = calculation_table[row_index][column_index - 1] = not (
                    calculation_table[row_index - 1][column_index]
                    and calculation_table[row_index - 1][column_index - 1]
                )  # NAND
                element = ""
        case "R":
            if last_element == "r":
                calculation_table[row_index][column_index] = calculation_table[row_index][column_index - 1] = not (
                    calculation_table[row_index - 1][column_index]
                )
                # NOT
                element = ""
        case "r":
            if last_element == "R":
                calculation_table[row_index][column_index] = calculation_table[row_index][column_index - 1] = not (
                    calculation_table[row_index - 1][column_index - 1]
                )
                # NOT
                element = ""
        case "B":
            if last_element == "B":
                calculation_table[row_index][column_index] = calculation_table[row_index - 1][column_index]
                calculation_table[row_index][column_index - 1] = calculation_table[row_index - 1][column_index - 1]
                # EQUAL
                element = ""

if element.startswith("L"):
    calculation_table[row_index][column_index] = calculation_table[row_index - 1][column_index]
    # EQUAL
return element
\end{lstlisting}
Diese Methode findet doppel Blöcke anhand des letzen Elements und wendet so die Blockfunktion an dem CALC\textunderscore TABLE an.\\
Für erfolgreich gefundene Blöcke wird das geradige Element auf einen leeren String gesetzt, damit die Methode WWWW,BBBB,... nicht als 3 Blöcke, sondern als 2 Blöcke erkennt.\\
Die Methode gibt dann das geradie Element zurück, damit es als letztes Element für das nächste Aufrufen der Funktion gespeichert werden kann.\\
\end{document}